\documentclass[10pt,a4paper]{moderncv}
\moderncvtheme[blue]{classic}                
\usepackage[utf8]{inputenc}
\usepackage[scale=0.8]{geometry}





\AtBeginDocument{\setlength{\maketitlenamewidth}{9cm}}
\firstname{Guillaume}
\familyname{Fontaine}
\title{Docteur en Mécanique des Fluides}              
\address{114 Chemin du Tamaye}{06560 Valbonne}    
\mobile{06 40 48 27 18}                    
\email{guillaume\_fontaine@hotmail.fr}                      
\extrainfo{32 ans, Permis B}

\begin{document}
\maketitle

\section{Diplômes et Études}
\cventry{\textbf{2012}}{Thèse de Doctorat, Université d'Aix Marseille -- Mécanique Des Fluides}{Université de Provence}{}{}{}
\cventry{\textbf{2008}}{Ingénieur Polytech' Marseille -- Mécanique Energétique}{Université de Provence}{\newline Energétique, Thermique, Mécanique des fluides}{}{Spécialité Matériaux}
\cventry{\textbf{2008}}{Master Recherche -- Mécanique, Physique et Modélisation}{Université de Provence}{}{}{Spécialité Energétique et Combustion}
\cventry{\textbf{2002-2005}}{Classes préparatoires MPSI et PSI}{}{}{}{}

\section{Experiences}
\cventry{\textbf{Depuis Juillet 2013}}{Ingénieur de Recherche -- Mécanique des Fluides}{K-Epsilon}{Sophia-Antipolis}{}{\newline
\itemize{
	\item{\textbf{Modélisation en mécanique des fluides et interactions fluide-structure, Développement de logiciels}}
	\item{\textbf{CFD :} simulations numériques par méthodes RANS sur plateformes HPC}
	\item{\textbf{Hydrodynamique et Mécanique des Structures \textit{Offshore} :} Etude des phénomènes de  \textit{Vortex Induced Vibration}}
	\item{\textbf{Hydrodynamique Navale :} Stabilité, mesures des performances des navires, automatisation des études (développement d'un "bassin d'essai numérique")}
	\item{\textbf{Energies Marines Renouvelables :} Conception, études d'éoliennes et d'hydroliennes, simulations de machines tournantes, optimisation des performances}
	\item{\textbf{Voile de compétition :} Optimisations d'hydrofoils et de profils épais, mise en place de modeleurs paramétriques de géométries}
	\item{\textbf{Interactions Fluide-Structure membranaires :} Etudes sur ballons dirigeables, parachutes, voiles}}}
\cventry{\textbf{2012}}{Thèse de Doctorat -- Mécanique des Fluides}{M2P2}{Marseille}{}{Développement d'une approche multidomaine pour la simulation numérique pseudospectrale d'écoulements tournants avec parois.\newline}
\cventry{\textbf{2008-2010}}{Enseignement}{IUT de Génie Thermique et Environnement}{Marseille}{}{TP et TD de thermodynamique, Mécanique des fluides, bases de données, programmation}{\newline}
\cventry{\textbf{2008}}{Stage de Master-recherche/Ingénieur}{IUSTI}{Marseille}{}{Etudes en tube à chocs de l'instabilité de Richtmyer-Meshkov\newline{}}
\cventry{\textbf{Divers}}{Marseille, Région parisienne}{}{}{}{Stagiaire de Recherche en soufflerie supersonique, IUSTI.\newline{}Professeur particulier de Mathématiques et de Physique, de la 4ème au Bac.}


\newpage
\section{Compétences}
\subsection{Informatique}
\cvcomputer{\textbf{CAO}}{Catia, Rhinoceros, OpenCascade}{\textbf{CFD}}{FineMarine, STAR-CCM+, OpenFoam, Fluent}
\cvcomputer{\textbf{Maillage}}{GMSH, Hexpress, cfMesh}{\textbf{Modélisation}}{Volumes Finis, Eléments finis, Méthodes spectrales, Méthodes potentielles}\newline
\cvcomputer{\textbf{Langages}}{Python, Fortran, C/C++, Matlab, Shell}{\textbf{Autre}}{Maitrise de \LaTeX. 
 Maitrise des OS à base GNU/Linux.}

\subsection{Langues}
\cvcomputer{\textbf{Anglais}}{courant (860 au TOEIC, 8 ans de pratique technique quotidienne)}{\textbf{Allemand}}{Notions}
\section{Publications Principales}
\itemize{
	\item{\textbf{An attempt to reduce the membrane effects in Richtmyer–Meshkov instability shock tube experiments} (G. Fontaine, C. Mariani, S. Martinez, D. Souffland)}
	\item{\textbf{Etude numérique des VIV sur un riser flexible} (G. Fontaine, C. Lothodé, D. Gross)}
	\item{\textbf{Multidomain extension of a divergence free pseudo-spectral algorithm for the DNS of wall-confined rotating flows} (G.Fontaine, E. Serre, S. Poncet)}}	
\section{Divers}
\cvline{Loisirs}{Windsurf, Kite-surf, Ski, Voile, Musique, Escalade, Informatique.}
\end{document}
