\documentclass[11pt]{article}

\usepackage{kantlipsum}   %% only for demo
\usepackage[utf8x]{inputenc}
\usepackage[T1]{fontenc}
\usepackage{lmodern}
\usepackage{marvosym}
\usepackage{graphicx}


\pagestyle{empty}
\usepackage{french}
\usepackage[scale=0.775]{geometry}
\setlength{\parindent}{0pt}
\addtolength{\parskip}{6pt}

\def\firstname{Guillaume}
\def\familyname{Fontaine}
\def\FileAuthor{\firstname \familyname}
\def\FileTitle{\firstname \familyname's cover letter}
\def\FileSubject{Cover letter}
\def\FileKeyWords{\firstname \familyname, Cover letter}

\renewcommand{\ttdefault}{pcr}

\usepackage{url}
\urlstyle{tt}
\ifpdf
  \usepackage[pdftex,pdfborder=0,breaklinks,baseurl=http://,pdfpagemode=None,pdfstartview=XYZ,pdfstartpage=1]{hyperref}
  \hypersetup{
    pdfauthor   = \FileAuthor,%
    pdftitle    = \FileTitle,%
    pdfsubject  = \FileSubject,%
    pdfkeywords = \FileKeyWords,%
    pdfcreator  = \LaTeX,%
    pdfproducer = \LaTeX}
\else
  \usepackage[dvips]{hyperref}
\fi


\begin{document}
\sffamily   % for use with a résumé using sans serif fonts;
%\rmfamily  % for use with a résumé using serif fonts;
\hfill%
\begin{minipage}[t]{.6\textwidth}
\raggedleft%
{\bfseries Guillaume Fontaine}\\[.35ex]
\small\itshape%
114 chemine de tameye\\
06560 Valbonne\\[.35ex]
\Telefon~0640482718\\
\Letter~\href{mailto:guillaume.fontai@gmail.com}{guillaume.fontai@gmail.com}
\end{minipage}\\[1em]
%
\begin{minipage}[t]{.4\textwidth}
\raggedright%
{\bfseries Thales Underwater Systems}\\[.35ex]
\small\itshape%
Sophia-Antipolis\\
\end{minipage}
\hfill % US style
%\\[1em] % UK style
\begin{minipage}[t]{.4\textwidth}
\raggedleft % US style
\today
%April 6, 2006 % US informal style
%05/04/2006 % UK formal style
\end{minipage}\\[2em]
%\raggedright
Madame, Monsieur,

Ingénieur-Docteur en mécanique des fluides, je suis à la recherche de nouvelles opportunités. Je suis salarié de l'entreprise K-Epsilon à Sophia-Antipolis, où j'exerce la fonction d'Ingénieur de Recherche en Hydrodynamique depuis juillet 2013.

A ce titre, je suis en charge de la gestion de projets de R\&D, de modélisation et de développement-logiciel, ainsi que des prestations industrielles en relation étroite avec le client. Je réalise des études en interaction fluide-structure, notamment sur les phénomènes de \textit{vibration induite par vortex} (VIV), qui m'ont permis d'établir un protocole de comparaison entre des simulations numériques en interaction fluide-structure et des campagnes expérimentales menées en bassin d'essai. J'étudie aussi le cas des structures très légères, caractérisées par des couplages très forts comme les voiles de course, kites, parachutes ou plus récemment des ballons dirigeables. C'est à travers ces différents projets que j'ai acquis une expérience très diversifiée en hydrodynamique et en interactions fluide-structure.

Mon activité m'a apporté une grande autonomie en développement et en modélisation : j'ai en effet réalisé des calculs de CFD complexes à travers des solutions industrielles et des outils créés au sein de la société. J'ai ainsi été amené à développer un bassin d'essais numériques pour la résistance à l'avancement des navires, et des solutions d'optimisation semi-automatiques pour des hydroliennes, des propulseurs marins ou des hydrofoils. Je participe depuis quelques mois au développement et à l'industrialisation du logiciel d'interaction fluide-structure de la société, en vue de sa commercialisation.

Etant passionné de sciences-physiques en général, je cherche aujourd'hui à diversifier mon activité professionnelle. J'ai étudié avec intérêt les vibrations induites par le fluide sur des structures, et la modélisation des phénomènes acoustiques en milieu marin éveille ma curiosité par la diversité de conditions environnementales à considérer. L'intégration des technologies associées dans une industrie de pointe en font à mes yeux un excellent défi. Ainsi, la description du poste « Ingénieur étude propagation acoustique sous-marine et calcul de performances SONAR » semble en adéquation avec ma volonté d'aborder les phénomènes fluides souvel angle, autre que celui de l'hydrodynamique. 

Dans l’espoir que ces quelques éléments soient de nature à vous faire considérer avec bienveillance ma candidature, je vous prie d’agréer Madame, Monsieur, l’expression de mes salutations distinguées.

%Yours sincerely,\\[2em] % if the opening is "Dear Mr(s) Doe,"
Cordialement,\\[2em] % if the opening is "Dear Sir or Madam,"
%
%\includegraphics[scale=0.75]{signature_blue}\\
{\bfseries Guillaume Fontaine}\\
%

\vfill%
\end{document}
